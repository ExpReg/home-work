\documentclass[a4paper]{article}
\usepackage[utf8]{inputenc}

\begin{document}
\author{Mads Riis Jensen (madjen12@student.aau.dk)}
\title{PP-mini-project}
\maketitle

\section*{Status}
\begin{itemize}
\item \textbf{Completeness:} Entire exercise.
\item \textbf{Runnable:} All functions are runnable, however there are some that assume that input are on the correct form so given wrong input they may not give the output wanted. The HTML calendar is given in the form of a string so no files are written.   
\item \textbf{Works as prescribed:} All functions work return output as prescribed in the exercise text(assuming input on the right form).
\item \textbf{version:} Im using Racket v5.3.6. with the Drracket editor(\#lang rakcet) 
\end{itemize}

\section{Overview of calendar language}

This section provides an overview of the scheme calendar language.

\subsection{Time}

Time is represented as the list ('time,Year,Month,Day,Hour,Min(minutes)),where 'time is a symbol used as identify the list as Time and the rest are all integers. A Time is created by the function \textit{createTime(Year,Month,Day,Hour,Min)}. The function checks whether the user has entered legal values, however it does not check whether the user inserted integers. Time also have a number of associated comparison functions, such as \textit{before (time1 time2)} which returns true if time1 is before time2. Most comparison functions are generated by the function \textit{generic-timeCheck(pred)}. Lastly Time has some functions that are mainly used in the presentation part,such as \textit{hour-min->string(time)} which returns the hour and minutes as the string "hour:min"(12:15) 

\subsection{Appointment}
An Appointment is represented as a list ('appointment,text,startTime,endTime) where 'appointment is a symbol,text is a string and startTime and endTime are both Time. Appointment also have a number of associated comparison functions e.g. \textit{appointment-before(app1 app2)}, which checks whether app1 starts before app2. Only the start-times of appointments are compared. 

\subsection{Calendar}
A calendar is a list of Appointments and calendars. There is functions to add or remove an appointment from a calendar. The remove-function, called \textit{removeFromCal(cal . toRemove)} removes whatever is in toRemove from the Calendar cal and any Calendar that cal might contain. Calendar has a number of functions that finds appointments that specify some predicate. Two of these \textit{find-last-appointment(cal pred)} and \textit{find-first-appointment(cal pred)} are both made from \textit{generic-finder(version)}. 

\subsection{Presentation and HTML}
To easily create HTML tags i created the function \textit{tag-creator(head)} where head is the name of a tag. \textit{tag-creator} returns a function that takes an arbitrary amount of either attributes,strings or other tag functions and puts them into a list. Attributtes are on the form 'symbol "value" ,whereas the content in the tags can either be strings or other tag-procedures.

The \textit{present-calendar-html(cal from-time to-time)} is build up of a number of smaller functions. 


\end{document}